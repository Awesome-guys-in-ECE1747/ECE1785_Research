\documentclass[]{IEEEphot}

\jvol{1}
\jnum{1}
% \jmonth{June}
% \pubyear{2009}

\newtheorem{theorem}{Theorem}
\newtheorem{lemma}{Lemma}

%Think about a research idea that related to your own area and you need to collect information from stakeholders through interview. (see examples above)
%1) decide what your purpose is and write a sentence describing it.
%2) develop an interview protocol. The protocol can be short, focusing on exactly what you are interested in. You should anticipate short interviews, perhaps 15-20 minutes at most.
%3) conduct two interviews.
%4) be prepared to tell the class what you learned, how the interviews went, any problems or lessons you can share. In future classes, we will learn more structured ways of analyzing qualitative data such as interview transcripts.


\begin{document}

\title{Assignment 2: Literature Review and Theory}

\author{Wenrui Xu}

\affil{Assignment of ECE1785H 2022W}  

% \doiinfo{DOI: 10.1109/JPHOT.2009.XXXXXXX\\
% 1943-0655/\$25.00 \copyright 2009 IEEE}%

\maketitle

\markboth{Assignment 2}{Literature Review and Theory}

% \begin{receivedinfo}%
% Manuscript received March 3, 2008; revised November 10, 2008. First published December 10, 2008. Current version published February 25, 2009. This research was sponsored by the National Science Foundation through the NSF ERC Center for Extreme Ultraviolet Science and Technology, NSF Award No. EEC-0310717. This paper was presented in part at the National Science Foundation.
% \end{receivedinfo}

% \begin{abstract}
% Three dimensional images were obtained using a single high numerical aperture hologram recorded in a high resolution photoresist with a table top $\alpha = 46.9$ nm laser. Gabor holograms numerically reconstructed over a range of image planes by sweeping the propagation distance allow numerical optical sectioning that results in a robust three dimension image of a test object with a resolution in depth of approximately and a lateral resolution of 164 nm. 
% \end{abstract}

% \begin{IEEEkeywords}
% Holography, image analysis.
% \end{IEEEkeywords}

% \section{Introduction}

% Conc:weakness of each method

%% \ackrule

\bibliographystyle{IEEEtran}
\bibliography{thesis}

%\section*{Biographies}

%\textbf{P. W. Wachulak} received the degree${\ldots}$ \\[6pt]
%\textbf{M. C. Marconi} received the degree${\ldots}$ \\[6pt]
%\textbf{R. A. Bartels} received the degree${\ldots}$ \\[6pt]
%\textbf{C. S. Menoni} received the degree${\ldots}$ \\[6pt]
%\textbf{J. J. Rocca} received the degree${\ldots}$

\begin{thebibliography}{99}  

\bibitem{ref1}Xiao, S., Witschey, J., \& Murphy-Hill, E. (2014). Social influences on secure development tool adoption: why security tools spread  Download Social influences on secure development tool adoption: why security tools spread, Proceedings of the 17th ACM conference on Computer supported cooperative work \& social computing (pp. 1095-1106): ACM.
\bibitem{ref2}Mokyr, J. (2002). The Gifts of Athena: Historical Origins of the Knowledge Economy.  Download The Gifts of Athena: Historical Origins of the Knowledge Economy.Princeton, NJ: Princeton University Press. (Ch 1).
\bibitem{ref3}Kahneman, D. (2011). Thinking, Fast and Slow. New York, NY: Farrar, Straus and Giroux. Ch. 12  Download Thinking, Fast and Slow. New York, NY: Farrar, Straus and Giroux. Ch. 12, The science of availability.
\bibitem{ref4}Argote, L., \& Ren, Y. (2012). Transactive memory systems: A microfoundation of dynamic capabilities.  Download Transactive memory systems: A microfoundation of dynamic capabilities. Journal of Management Studies, 49(8), 1375-1382.
\bibitem{ref5}Donath, J. (2008). Signals in social supernets  Download Signals in social supernets. Journal of Computer Mediated Communication, 13(1), 231-251.

\end{thebibliography}

\end{document}