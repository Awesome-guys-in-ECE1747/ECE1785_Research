\documentclass[]{IEEEphot}
\usepackage{cite}
\jvol{1}
\jnum{1}
% \jmonth{June}
% \pubyear{2009}
\usepackage{listings}
\usepackage{ctex}

% 用来设置附录中代码的样式

% \lstset{
%     basicstyle          =   \sffamily,          % 基本代码风格
%     keywordstyle        =   \bfseries,          % 关键字风格
%     commentstyle        =   \rmfamily\itshape,  % 注释的风格,斜体
%     stringstyle         =   \ttfamily,  % 字符串风格
%     flexiblecolumns,                % 别问为什么,加上这个
%     numbers             =   left,   % 行号的位置在左边
%     showspaces          =   false,  % 是否显示空格,显示了有点乱,所以不现实了
%     numberstyle         =   \zihao{-5}\ttfamily,    % 行号的样式,小五号,tt等宽字体
%     showstringspaces    =   false,
%     captionpos          =   t,      % 这段代码的名字所呈现的位置,t指的是top上面
%     frame               =   lrtb,   % 显示边框
% }

% \lstdefinestyle{Python}{
%     language        =   Python, % 语言选Python
%     basicstyle      =   \zihao{-5}\ttfamily,
%     numberstyle     =   \zihao{-5}\ttfamily,
%     keywordstyle    =   \color{blue},
%     keywordstyle    =   [2] \color{teal},
%     stringstyle     =   \color{magenta},
%     commentstyle    =   \color{red}\ttfamily,
%     breaklines      =   true,   % 自动换行,建议不要写太长的行
%     columns         =   fixed,  % 如果不加这一句,字间距就不固定,很丑,必须加
%     basewidth       =   0.5em,
% }
\newtheorem{theorem}{Theorem}
\newtheorem{lemma}{Lemma}

%Think about a research idea that related to your own area and you need to collect information from stakeholders through interview. (see examples above)
%1) decide what your purpose is and write a sentence describing it.
%2) develop an interview protocol. The protocol can be short, focusing on exactly what you are interested in. You should anticipate short interviews, perhaps 15-20 minutes at most.
%3) conduct two interviews.
%4) be prepared to tell the class what you learned, how the interviews went, any problems or lessons you can share. In future classes, we will learn more structured ways of analyzing qualitative data such as interview transcripts.


\begin{document}

\title{Assignment 3: Intro Quantitative Analysis}

\author{Jiaming Xu}

\affil{Assignment of ECE1785H 2022W}  

% \doiinfo{DOI: 10.1109/JPHOT.2009.XXXXXXX\\
% 1943-0655/\$25.00 \copyright 2009 IEEE}%

\maketitle

\markboth{Assignment 3}{Intro Quantitative Analysis}

\section{Introduction \& Motivation}
As part of our final project, I have been doing interviews with many software engineers in big companies.
To my surprise, many of them have contributed or are making contributions to the open-source projects. 
So I'm curious about what's the big companies attitude towards these open-source projects related experience.\par 
Do we need to make more contributions to open-source projects if we want to apply for some companies which flavor open-source communities? 
Does age have an impact within? Does young people have to contribute more to open-source code to get their jobs?
Will the employers ask us to contribute to open-source communities as our job if they encourage using/contributing to open-source projects?
I think this problem matters because it can help me and many other software engineering students 
have a clearer view of this aspect before we try to seek for jobs. Hopefully, \cite{GitHubOpenSourceSurvey2017}\cite{DBLP:journals/corr/Geiger17} has mentioned part of these questions and data, so this report will explore these questions.


\section{Research Questions}
\begin{itemize}
    \item RQ1: Do young people have to contribute more to open-source projects to get their position in companies which flavor open-source communities?\par
    \item RQ2: Do employees tend to contribute to open-source projects as their job in companies which encourage using to open-source projects?
\end{itemize}

\section{Data Cleaning \& Analysis}
This is the dataset\cite{GitHubOpenSourceDataset2017}\. I used 'Python' to do the data cleaning and analysis.
To do the data cleaning, I use the 'pandas' lib: the read\_csv function to read the data and dropna function to drop the rows which has a NaN data.\par
For the first question, I choose the 'AGE' column to represent the age and working experience of the engineers, the 'EMPLOYER.POLICY.APPLICATIONS' column to show whether the companies flavor the open-source communities and the 'OSS.HIRING' column to explore if the experience helps them get their jobs.\par
For the second question, I choose the 'OSS.AS.JOB' column to represent whether they tend to contribute to open-source projects as their job, 'EMPLOYER.POLICY.APPLICATIONS' column to show their employers attitude towards open-source communities.
In order for the data to fit into the linear regression model, I transformed the data into numbers. You can find the code in the appendix.\par
'OSS.HIRING': 'Very important':1,'Somewhat important':2,'Not at all important':3,'Not too important':4,'Not applicable-I hadn\'t made any contributions when I got this job.':5\par
'AGE': '17 or younger':1,'18 to 24 years':2,'25 to 34 years':3,'35 to 44 years':4,'45 to 54 years':5,'55 to 64 years':6,'65 years or older':7\par
'EMPLOYER.POLICY.APPLICATIONS':'Use of open source applications is encouraged.':1,'Use of open source applications is acceptable if it is the most appropriate tool.':2,'My employer doesn\'t have a clear policy on this.':3,'Not applicable':4,'I\'m not sure.':5,'Use of open source applications is rarely, if ever, permitted.':6\par
'OSS.AS.JOB'='Yes, directly-  some or all of my work duties include contributing to open source projects.':1,'Yes, indirectly- I contribute to open source in carrying out my work duties, but I am not required or expected to do so.':2,'No.':3
    
\section{Results}
\subsection{RQ1}
The formula is $y=0.04301915x_{1}+0.23456672x_{2}+c$ where $y$ is the predicted 'OSS.HIRING', $x_{1}$ is 'AGE', $x_{2}$ is 'EMPLOYER.POLICY.APPLICATIONS', c is a constant number.
\begin{table}[]
    \begin{tabular}{|l|l|l|}
    \hline
    x1 & x2 & y                  \\ \hline
    1  & 1  & 2.293055827077731  \\ \hline
    1  & 2  & 2.5276225515024704 \\ \hline
    1  & 3  & 2.76218927592721   \\ \hline
    1  & 4  & 2.9967560003519496 \\ \hline
    1  & 5  & 3.231322724776689  \\ \hline
    1  & 6  & 3.4658894492014287 \\ \hline
    2  & 1  & 2.3360749754228634 \\ \hline
    2  & 2  & 2.5706416998476027 \\ \hline
    2  & 3  & 2.805208424272342  \\ \hline
    2  & 4  & 3.039775148697082  \\ \hline
    2  & 5  & 3.2743418731218217 \\ \hline
    2  & 6  & 3.5089085975465606 \\ \hline
    3  & 1  & 2.3790941237679952 \\ \hline
    3  & 2  & 2.613660848192735  \\ \hline
    3  & 3  & 2.8482275726174744 \\ \hline
    3  & 4  & 3.0827942970422137 \\ \hline
    3  & 5  & 3.3173610214669536 \\ \hline
    3  & 6  & 3.551927745891693  \\ \hline
    4  & 1  & 2.4221132721131275 \\ \hline
    4  & 2  & 2.656679996537867  \\ \hline
    4  & 3  & 2.8912467209626067 \\ \hline
    4  & 4  & 3.125813445387346  \\ \hline
    4  & 5  & 3.3603801698120854 \\ \hline
    4  & 6  & 3.5949468942368252 \\ \hline
    5  & 1  & 2.46513242045826   \\ \hline
    5  & 2  & 2.6996991448829992 \\ \hline
    5  & 3  & 2.9342658693077386 \\ \hline
    5  & 4  & 3.1688325937324784 \\ \hline
    5  & 5  & 3.4033993181572177 \\ \hline
    5  & 6  & 3.637966042581957  \\ \hline
    6  & 1  & 2.5081515688033917 \\ \hline
    6  & 2  & 2.7427182932281315 \\ \hline
    6  & 3  & 2.977285017652871  \\ \hline
    6  & 4  & 3.2118517420776103 \\ \hline
    6  & 5  & 3.44641846650235   \\ \hline
    6  & 6  & 3.6809851909270894 \\ \hline
    7  & 1  & 2.551170717148524  \\ \hline
    7  & 2  & 2.7857374415732634 \\ \hline
    7  & 3  & 3.020304165998003  \\ \hline
    7  & 4  & 3.2548708904227426 \\ \hline
    7  & 5  & 3.489437614847482  \\ \hline
    7  & 6  & 3.7240043392722217 \\ \hline
    \end{tabular}
    \end{table}
As is shown above, the less your companies show positive attitude to open-source communities, the older you are, your open-source experience is less likely to help you get the job.
Controlling the age $x_{1}$ to be the same, we can see that $y$ is increasing with $x_{2}$. It means the companies which do not like open-source projects pay less attention to the related working experience when hiring engineers.
Controlling the policy $x_{2}$ to be the same, we can see that $y$ is increasing with $x_{1}$. It means the older you are, the open-source projects will help you less to get your job.
\subsection{RQ2}
    The formula is $y=0.20563629x+c$ where $y$ is the predicted 'OSS.AS.JOB', $x$ is 'EMPLOYER.POLICY.APPLICATIONS',c is a constant number.
[[0.20563629]]
\begin{table}[]
    \begin{tabular}{|l|l|}
    \hline
 x & y                  \\ \hline
 1  & 1.9315087682434604  \\ \hline
 2  & 2.1371450584278575 \\ \hline
 3  & 2.342781348612254   \\ \hline
 4  & 2.5484176387966513 \\ \hline
 5  & 2.7540539289810484 \\ \hline
 6  & 2.959690219165445 \\ \hline
\end{tabular}
\end{table}

 The result shows that if your company has a positive attitude to open-source projects, you are more likely to contribute to open-source project as your paid job or part of it.
\section{Conclusion}
In conclusion, if we want to enter a company, it helps a lot to know whether this company flavors the open-source projects. If it does, we should contribute more to the open-source projects which can help us greatly in the application.
And as young students, we should do more open-source contributions for our future jobs.
What's more, in companies which have a positive attitude towards open-source projects, we are more likely to contribute to open-source projects as our job.
% \begin{receivedinfo}%
% Manuscript received March 3, 2008; revised November 10, 2008. First published December 10, 2008. Current version published February 25, 2009. This research was sponsored by the National Science Foundation through the NSF ERC Center for Extreme Ultraviolet Science and Technology, NSF Award No. EEC-0310717. This paper was presented in part at the National Science Foundation.
% \end{receivedinfo}

% \begin{abstract}
% Three dimensional images were obtained using a single high numerical aperture hologram recorded in a high resolution photoresist with a table top $\alpha = 46.9$ nm laser. Gabor holograms numerically reconstructed over a range of image planes by sweeping the propagation distance allow numerical optical sectioning that results in a robust three dimension image of a test object with a resolution in depth of approximately and a lateral resolution of 164 nm. 
% \end{abstract}

% \begin{IEEEkeywords}
% Holography, image analysis.
% \end{IEEEkeywords}

% \section{Introduction}

% Conc:weakness of each method

%% \ackrule

\bibliographystyle{IEEEtran}
\bibliography{template}

% \appendix
% \section{Code}
% \begin{lstlisting}
%     # This is a sample Python script.

% # Press Shift+F10 to execute it or replace it with your code.
% # Press Double Shift to search everywhere for classes, files, tool windows, actions, and settings.
% import pandas as pd
% from sklearn.linear_model import LinearRegression as lr
% def data_cleaning(data):
%     # print()
%     # Use a breakpoint in the code line below to debug your script.
%     # print(f'Hi, {name}')  # Press Ctrl+F8 to toggle the breakpoint.
%     data1=data.loc[:,['OSS.HIRING','EMPLOYER.POLICY.APPLICATIONS','AGE']].dropna()
%     # print(data1.describe())
%     # print("which has Nan?\n", data1.isnull().sum(), "\n")
%     hiring={'Very important':1,'Somewhat important':2,'Not at all important':3,'Not too important':4,'Not applicable-I hadn\'t made any contributions when I got this job.':5}
%     age={'17 or younger':1,'18 to 24 years':2,'25 to 34 years':3,'35 to 44 years':4,'45 to 54 years':5,'55 to 64 years':6,'65 years or older':7}
%     policy={'Use of open source applications is encouraged.':1,'Use of open source applications is acceptable if it is the most appropriate tool.':2,'My employer doesn\'t have a clear policy on this.':3,'Not applicable':4,'I\'m not sure.':5,'Use of open source applications is rarely, if ever, permitted.':6}
%     # policy={'I am free to contribute without asking for permission.':1,'My employer doesn\'t have a clear policy on this.':2,'I am permitted to contribute to open source, but need to ask for permission.':3,'I\'m not sure.':4,'Not applicable':5,'I am not permitted to contribute to open source at all.':6}
%     job={'Yes, directly-  some or all of my work duties include contributing to open source projects.':1,'Yes, indirectly- I contribute to open source in carrying out my work duties, but I am not required or expected to do so.':2,'No.':3}
%     data1['OSS.HIRING']=data1['OSS.HIRING'].map(hiring)
%     data1['EMPLOYER.POLICY.APPLICATIONS'] = data1['EMPLOYER.POLICY.APPLICATIONS'].map(policy)
%     data1['AGE'] = data1['AGE'].map(age)
%     data2=data.loc[:,['OSS.AS.JOB','EMPLOYER.POLICY.APPLICATIONS']].dropna()
%     data2['OSS.AS.JOB']=data2['OSS.AS.JOB'].map(job)
%     data2['EMPLOYER.POLICY.APPLICATIONS'] = data2['EMPLOYER.POLICY.APPLICATIONS'].map(policy)
%     # print("which has Nan?\n", data2.isnull().sum(), "\n")
%     return data1,data2

% def data1_analysis(data):
%     # print(data.describe())
%     # print(data.head(5))
%     lrmodel=lr()
%     # print("which has Nan?\n", data.isnull().sum(), "\n")
%     data[['AGE','EMPLOYER.POLICY.APPLICATIONS','OSS.HIRING']].corr()
%     x=data[['AGE','EMPLOYER.POLICY.APPLICATIONS']]
%     y=data[['OSS.HIRING']]
%     lrmodel.fit(x,y)
%     print(lrmodel.coef_)
%     for i in range(7):
%         for j in range(6):
%             print(i+1,j+1,lrmodel.predict([[i+1,j+1]])[0][0])
%     # print(lrmodel.predict([[2,1]]))

% def data2_analysis(data):
%     lrmodel=lr()
%     data[['EMPLOYER.POLICY.APPLICATIONS','OSS.AS.JOB']].corr()
%     x=data[['EMPLOYER.POLICY.APPLICATIONS']]
%     y=data[['OSS.AS.JOB']]
%     lrmodel.fit(x,y)
%     print(lrmodel.coef_)
%     for i in range(6):
%         print(i + 1, lrmodel.predict([[i + 1]])[0][0])
%     print('1')

% # Press the green button in the gutter to run the script.
% if __name__ == '__main__':
%     # print_hi('PyCharm')
%     data=pd.read_csv('./survey_data.csv')
%     data1,data2=data_cleaning(data)
%     data1_analysis(data1)
%     data2_analysis(data2)
% # See PyCharm help at https://www.jetbrains.com/help/pycharm/
% \end{lstlisting}
%\section*{Biographies}

%\textbf{P. W. Wachulak} received the degree${\ldots}$ \\[6pt]
%\textbf{M. C. Marconi} received the degree${\ldots}$ \\[6pt]
%\textbf{R. A. Bartels} received the degree${\ldots}$ \\[6pt]
%\textbf{C. S. Menoni} received the degree${\ldots}$ \\[6pt]
%\textbf{J. J. Rocca} received the degree${\ldots}$

% \bibliographystyle{ACM-Reference-Format}
% \bibliography{sample-base}
% \begin{thebibliography}{99}  

% \bibitem{ref1}Xiao, S., Witschey, J., \& Murphy-Hill, E. (2014). Social influences on secure development tool adoption: why security tools spread  Download Social influences on secure development tool adoption: why security tools spread, Proceedings of the 17th ACM conference on Computer supported cooperative work \& social computing (pp. 1095-1106): ACM.
% \bibitem{ref2}Mokyr, J. (2002). The Gifts of Athena: Historical Origins of the Knowledge Economy.  Download The Gifts of Athena: Historical Origins of the Knowledge Economy.Princeton, NJ: Princeton University Press. (Ch 1).
% \bibitem{ref3}Kahneman, D. (2011). Thinking, Fast and Slow. New York, NY: Farrar, Straus and Giroux. Ch. 12  Download Thinking, Fast and Slow. New York, NY: Farrar, Straus and Giroux. Ch. 12, The science of availability.
% \bibitem{ref4}Argote, L., \& Ren, Y. (2012). Transactive memory systems: A microfoundation of dynamic capabilities.  Download Transactive memory systems: A microfoundation of dynamic capabilities. Journal of Management Studies, 49(8), 1375-1382.
% \bibitem{ref5}Donath, J. (2008). Signals in social supernets  Download Signals in social supernets. Journal of Computer Mediated Communication, 13(1), 231-251.

% \end{thebibliography}

\end{document}