%%
%% This is file `sample-acmlarge.tex',
%% generated with the docstrip utility.
%%
%% The original source files were:
%%
%% samples.dtx  (with options: `acmlarge')
%% 
%% IMPORTANT NOTICE:
%% 
%% For the copyright see the source file.
%% 
%% Any modified versions of this file must be renamed
%% with new filenames distinct from sample-acmlarge.tex.
%% 
%% For distribution of the original source see the terms
%% for copying and modification in the file samples.dtx.
%% 
%% This generated file may be distributed as long as the
%% original source files, as listed above, are part of the
%% same distribution. (The sources need not necessarily be
%% in the same archive or directory.)
%%
%%
%% Commands for TeXCount
%TC:macro \cite [option:text,text]
%TC:macro \citep [option:text,text]
%TC:macro \citet [option:text,text]
%TC:envir table 0 1
%TC:envir table* 0 1
%TC:envir tabular [ignore] word
%TC:envir displaymath 0 word
%TC:envir math 0 word
%TC:envir comment 0 0
%%
%%
%% The first command in your LaTeX source must be the \documentclass command.
\documentclass[acmlarge]{acmart}

%%
%% \BibTeX command to typeset BibTeX logo in the docs
\AtBeginDocument{%
  \providecommand\BibTeX{{%
    \normalfont B\kern-0.5em{\scshape i\kern-0.25em b}\kern-0.8em\TeX}}}

%% Rights management information.  This information is sent to you
%% when you complete the rights form.  These commands have SAMPLE
%% values in them; it is your responsibility as an author to replace
%% the commands and values with those provided to you when you
%% complete the rights form.
% \setcopyright{acmcopyright}
% \copyrightyear{2022}
% \acmYear{2022}
% \acmDOI{XXXXXXX.XXXXXXX}


%%
%% These commands are for a JOURNAL article.
\acmJournal{POMACS}
% \acmVolume{37}
% \acmNumber{4}
% \acmArticle{111}
\acmMonth{3}

%%
%% Submission ID.
%% Use this when submitting an article to a sponsored event. You'll
%% receive a unique submission ID from the organizers
%% of the event, and this ID should be used as the parameter to this command.
%%\acmSubmissionID{123-A56-BU3}

%%
%% The majority of ACM publications use numbered citations and
%% references.  The command \citestyle{authoryear} switches to the
%% "author year" style.
%%
%% If you are preparing content for an event
%% sponsored by ACM SIGGRAPH, you must use the "author year" style of
%% citations and references.
%% Uncommenting
%% the next command will enable that style.
%%\citestyle{acmauthoryear}

%%
%% end of the preamble, start of the body of the document source.
\begin{document}

%%
%% The "title" command has an optional parameter,
%% allowing the author to define a "short title" to be used in page headers.
\title{Emotion Impacts in Coding Efficiency}

%%
%% The "author" command and its associated commands are used to define
%% the authors and their affiliations.
%% Of note is the shared affiliation of the first two authors, and the
%% "authornote" and "authornotemark" commands
%% used to denote shared contribution to the research.

\author{Jiaming Xu}
\email{jm.xu@mail.utoronto.ca}
\affiliation{%
  \institution{1007698831, Group 8, ECE, University of Toronto}
  \city{Toronto}
  \state{Ontario}
  \country{Canada}
  \postcode{M5B 0A5}
}

\author{Teng Yue}
\email{larst@affiliation.org}
\affiliation{% 
  \institution{1007826792, Group 8, ECE, University of Toronto}
  \city{Toronto}
  \state{Ontario}
  \country{Canada}
  \postcode{M5B 0A5}}


\author{Wenrui Xu}
\email{larst@affiliation.org}
\affiliation{%
  \institution{1008313228, Group 8, ECE, University of Toronto}
  \city{Toronto}
  \state{Ontario}
  \country{Canada}
  \postcode{M5B 0A5}
}
%%
%% By default, the full list of authors will be used in the page
%% headers. Often, this list is too long, and will overlap
%% other information printed in the page headers. This command allows
%% the author to define a more concise list
%% of authors' names for this purpose.
% \renewcommand{\shortauthors}{Trovato and Tobin, et al.}

%%
%% The abstract is a short summary of the work to be presented in the
%% article.
\begin{abstract}
  This report mainly discussed the impact of different emotions may have on coding efficiency. We use both quantitative and qualitative analysis to find the results. For the quantitative analysis, we investigate the open-source repositories, mine the emotion hidden inside the commit messages and find the relationship between emotion and coding efficiency. For the qualitative analysis, we make some interviews with software engineers from big companies, research teams and open-source community.

\end{abstract}

%%
%% The code below is generated by the tool at http://dl.acm.org/ccs.cfm.
%% Please copy and paste the code instead of the example below.
%%
% \begin{CCSXML}
% <ccs2012>
%  <concept>
%   <concept_id>10010520.10010553.10010562</concept_id>
%   <concept_desc>Computer systems organization~Embedded systems</concept_desc>
%   <concept_significance>500</concept_significance>
%  </concept>
%  <concept>
%   <concept_id>10010520.10010575.10010755</concept_id>
%   <concept_desc>Computer systems organization~Redundancy</concept_desc>
%   <concept_significance>300</concept_significance>
%  </concept>
%  <concept>
%   <concept_id>10010520.10010553.10010554</concept_id>
%   <concept_desc>Computer systems organization~Robotics</concept_desc>
%   <concept_significance>100</concept_significance>
%  </concept>
%  <concept>
%   <concept_id>10003033.10003083.10003095</concept_id>
%   <concept_desc>Networks~Network reliability</concept_desc>
%   <concept_significance>100</concept_significance>
%  </concept>
% </ccs2012>
% \end{CCSXML}

% \ccsdesc[500]{Computer systems organization~Embedded systems}
% \ccsdesc[300]{Computer systems organization~Redundancy}
% \ccsdesc{Computer systems organization~Robotics}
% \ccsdesc[100]{Networks~Network reliability}

%%
%% Keywords. The author(s) should pick words that accurately describe
%% the work being presented. Separate the keywords with commas.
\keywords{emotion mining, coding efficiency}


%%
%% This command processes the author and affiliation and title
%% information and builds the first part of the formatted document.
\maketitle

\section{Introduction}
Briefly introduce the research topic, including the background information, the motivation and contribution of our research.

\section{Literature Review}
Detailed information of definition of emotions, categorization and emotional mining.\par 
Major references are:\cite{c.elmohamed},\cite{ekman1999basic},\cite{wikiemotion},\cite{10.1115/1.4047685},\cite{shaver1987emotion},\cite{5693422},\cite{barsade2007does},\cite{pervez2010impact},\cite{coviello2020effect},\cite{kadoya2020emotional}.
\subsection{Definition \& Categories of Emotions}
The definition and features of emotions from a psychological aspect. Using a consistent categorization standard to analyze the emotions. Explain the categories and the words which belongs to these emotions.
In this part, we will discuss definition from wikipedia\cite{wikiemotion}, the Ekman emotion framework\cite{ekman1999basic}.
\subsection{Emotion and productivity}
In this part we will discuss how does emotion affect productivity. 
The papers are \cite{kadoya2020emotional},\cite{coviello2020effect},\cite{pervez2010impact},\cite{barsade2007does},\cite{shaver1987emotion}
\subsection{Emotional Mining}
Introduce the emotional mining methods used in github.The papers discussed in this part is \cite{5693422},\cite{10.1115/1.4047685},\cite{c.elmohamed}.

\section{Research Questions}
\begin{itemize}
  \item How do we define and categorize emotions?
  \item How do we know engineers’ emotions from their codes?
  \item How do we measure the efficiency of coding?
  \item Does the emotion change have impact on the efficiency of coding?
  \item What effect will the emotion change cause on coding efficiency?
\end{itemize}


\section{Methodology}
We will collect our data in both qualitative and quantitative ways.
\subsection{Quatitative Methods}
We will use the emotional mining methods to mine the emotions in the selected repositories which are easy to analyze.
Then we will calculate the coding efficiency of each mined commit messages accordingly.
\subsection{Qualitative Methods}
We designed an interview to make qualitative study on different groups of software engineers, including researchers in Universities, software engineers in large companies and open-source developers.
\section{Results}
This part will show the results of both quantitative and qualitative research methods.
\subsection{Quatitative Results}
This part will show the results of emotional mining.
\subsection{Qualitative Results}
This part will show the results of interviews.
\section{Discussion}
We will analyze the results of the study to show the relationship between different emotions and coding efficiency in both qualitative and quantitative ways.

\section{Conclusion}
Conclude the paper by summarizing the findings of the research questions and our contributions.
Then consider the future works to improve our researches.





%%
%% The next two lines define the bibliography style to be used, and
%% the bibliography file.
\bibliographystyle{ACM-Reference-Format}
\bibliography{sample-base}

%%
%% If your work has an appendix, this is the place to put it.
\appendix

\section{Critiques and Solutions}
Critique: "one concern that comes to my mind is that developers are often required to write commit messages in a certain format, which may not be able to convey emotions. For example, some short messages like ‘[merge] merge from main branch’ are neural and could be difficult to analyze. How would you cleanse the dataset while remaining unbiased is a challenge. Overall, the proposal was well structured and the research methods make sense to me.
"\par
Solution: We make a new design to select some open-source github repositories which contains relatively more valid data.
And we will pick some of them which are easy to analyze their emotions hidden inside, and abandon the others. We will reduce the bias by calculating the efficiency with the time between the selected commit log and its previous log. So we think this does not add much bias to our results.\par 

Critique: "Suggestions: - Your research topic is really interesting! Lots of great work is being done around the emotions of engineers/designers/developers. If you need some inspiration for how to analyze and process data related to emotions, here is a paper that investigates the emotions during mechanical design, that I think could be relevant:  https://doi.org/10.1115/1.4047685 - At times, the writing is vague or unclear. 
It may be more impactful for the statements/claims you make if you use examples and provide sources. 
- Try to reconsider using lines of code as your measure of working efficiency, 
considering it was used quite a few times in class as an example of a flawed metric. 
- Try to add more to your motivation for the next deliverables and really prove that what you are studying is important and impactful!
"\par
Solution: We read the paper and try to improve our writing by adding more examples. 
We think you made a good point about the measurement, we are considering using the coding value metrics as a backup option. And we will use some more examples to prove our study's significance.\par

Critique: "It is worth noting that generally speaking, the wording of the common commit logs is usually quite neutral. 
How to analyze the emotions of the programmers based on these commit messages accurately deserves further refinement. 
"\par
Solution: We refined our mining methods and try to mine the emotions from the commit messages using Regular Expression. We will pick the messages which contains the words expressing their feelings.\par


Critique: "I only have one concern that the fourth research question, ‘Does the emotion change have impact on the efficiency of coding?’ 
is a close-end question, maybe some further supplement to find some exact relationship or modifying it to be a ‘what’ question would be better? 
And I would suggest that combining qualitatively analyzing the coding efficiency via something like self-report of developers may also provide some help."\par
Solution: We added a new RQ to ask 'What impact the emotion change will have on the efficiency of coding?'. 
And we have changed part of our plans to include an interview of engineers/open-source developers.


Critique: "First, the use of interviews may generate biases, such as people who are willing to be interviewed tend to show a more positive mood when they are chosen and tracked, etc.
Also, you intend to analyze the programmer's commit messages to analyze the mood, but in reality, in many cases, the commit messages are probably just a very concise but objective description of the content of this code commit, not mixed with subjective emotions, so I think more considerations are needed to choose the quantitative analysis object.
"\par
Solution: We carefully select the interviewees instead of just recruit them with a reward. The interviewees are selected by us, not recruited with an advertisement. And we select them considering different factors including working experience, ages, job positions, etc. So I don't think there exist such problems.
As to the second point, we have mentioned it before.

Critique: "Probably need more literature reviews on previous studies for comparison"\par
Solution: We read more papers and references to add more literature reviews.\par

Critique: "Firstly, the research questions are specific, but the first three questions seem more about the background information instead of the analysis result of collected data. 
If it is possible to add more questions about the connection between emotion and code efficiency, like will engineers (students) try to switch their emotions when they try to promote efficiency? 
Is that useful? Do engineers realize which emotion is most optimal for their code efficiency? etc. 
Secondly, for the interview part, whether keeping a record of mood and code efficiency at the beginning of working will have an impact on their work efficiency? 
Since they may be more productive if they record their progress."\par
Solution: For the first problem, we think emotion is mostly an unconscious psychological activity, so most of time, people will not try to change it consciously. So we add some questions in our interview about the emotion changes and the coding efficiency changes to solve this problem.
For the second point, we revised our interview methods. We will not record their performance any more for two reasons: First, the time of research is too short to complete some projects. So it's meaningless to record their coding efficiency in a month. Second, as you said, if recorded, they will realize that they are monitored and increase their working efficiency unconsciously.
So we just use an interview to ask them recall their emotion change, working efficiency and evaluation during the project.
\par

Critique: "For the research questions, I think most of them are description questions, while other types of questions are relatively few. 
I suggest that more frequency and descriptive-process questions can be supplemented to the research questions. "\par
Solution: We revised the RQ to add more descriptive-process questions. It's a very good point. \par
\section{TODOs for the next 2 weeks}
Here is our plan of work for the next 2 weeks.
\begin{itemize}
  \item Keep interviewing more software engineers, including open-source coders, junior/senior engineers, PhD. students and researchers. 
  \item Get to know more about the emotion mining from the commit messages, and get the valid commit logs as the raw data.
  \item Select some appropriate open-source repositories which are easy to analyze as the study case.
  \item Record and translate the interview into English version to analyze the cases.  
\end{itemize}
\end{document}
\endinput
%%
%% End of file `sample-acmlarge.tex'.
